\section{Sample Data}

The game creates a timestamped log containing all chat messages that
the players send to each other, as well as all player actions (such as
dropping a block) and events that happen in the environment (such as
the ball being teleported). In addition, the game logs the position of
the players and other movable objects every 200 milliseconds. That
means, a full game can be re-played from the logged information.

We now present some excerpts of interactions captured during
playtesting which illustrate the kind of data that our game
elicits. Task oriented dialog contributions make up the majority of
the conversations. In particular, players give instructions to each
other, discuss strategy, and compare their maps. The following
excerpts illustrate these three types of subdialog. 

In Example
(\ref{ex:instructions}) player B explains to player A where to drop
some blocks. This example also shows some of the main strategies that
players use to identify locations. First, player B drops a block and
describes a location with respect to that block. At the end, player A
describes the ball's position using the playing area (\textit{``on top
  of the page''}) and an item in the background image (\textit{``above
  right nostril''}) as reference points. In a fourth strategy that we
have observed the players use the grid pattern implied by the blocks
used to build the walls to describe a location (e.g., \textit{``three
  over and two up''}).

{\footnotesize
\begin{example}
\parbox[t]{0.9\columnwidth}{
B: \textit{and THEN I need the wall of two blocks to the left...}

\strut[B drops block at (376,228)]

B: \textit{drop blocks BELOW the block I just dropped}

B: \textit{But only 2}

A: \textit{how far down?}

B: \textit{directly below, and then just below that}

\strut[A drops block at (372,248)]
\strut[A drops block at (372,264)]

B: \textit{perfect}

\strut[B pushes ball upwards into teleporter at (380, 80)]

B: \textit{Tadaaa}

\hangindent=1.3em A: \textit{i see a ball.  on top of page all the way to the top but
  above right nostril}
}
\label{ex:instructions}
\end{example}
}

In Example (\ref{ex:strategy}), player B is describing his current plan
to player A.

{\footnotesize
\begin{example}
\parbox[t]{0.9\columnwidth}{
B: \textit{So I need to get the ball into MY teleporter}

B: \textit{On the nose of the snake god.}

B: \textit{So you can see it, and then push it into the goal.}

B: \textit{Does that make sense?}

A: \textit{yes}
}
\label{ex:strategy}
\end{example}
}

Example (\ref{ex:comparing}) shows the players noticing a difference
in their maps. This is triggered by player A instructing player B to
drop a block in a location that is not accessible to player B. In
response, player B describes the layout of boundaries in his
map.  While task oriented dialog contributions account for the
majority of the chat dialog, players also exchange chitchat and spend
some time coordinating their communication, as shown at the end of
Example (\ref{ex:comparing}).


{\footnotesize
\begin{example}
\parbox[t]{0.9\columnwidth}{
\hangindent=1.3em A: \textit{so, i'll place a block and you place a block to the right of it.}

[A drops a block at (104,20)]

\hangindent=1.3em B: \textit{Ok. *I* have a boundary of blocks extending down from the top to the bottom right corner of  the tomb of the island king}

B: \textit{and then across to the dead mans cove}

\hangindent=1.3em B: \textit{That whole area, containing the ruins, is inaccesible to me.}

B: \textit{Don;t YOU hate Kris now too?}

A: \textit{:-)}

[A drops a block at (104,20)]

A: \textit{slow down, i need to read what you wrote.}

B: \textit{I type smaller words.}
}
\label{ex:comparing}
\end{example}
}
