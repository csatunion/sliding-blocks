\section{Sample Data}

We now present some excerpts of interactions we captured during
playtesting which illustrate the kind of data that our game
elicits. \todo{say that everything is timestamped; we capture
  movements, so that entire games will be replayable} The conversations consist mainly of the players giving
instructions to each other, discussing strategy, or comparing their
maps. In Example (\ref{ex:instructions}) player B explains to player A
where to drop some blocks. This example also shows some of the main
strategies that players use to identify locations. First, player B
drops a block and describes a location with respect to that block. At
the end, player A describes the ball's position using the playing area
(\textit{``on top of the page''}) and an item in the background image
(\textit{``above right nostril''}) as reference points. In a fourth
strategy that we have observed the players use the grid pattern implied
by the blocks used to build the walls to
describe a location (e.g., \textit{``three over and 2 up''}).

{\footnotesize
\begin{example}
\parbox[t]{0.9\columnwidth}{
B: \textit{and THEN I need the wall of two blocks to the left...}

\strut[B drops block at (376,228)]

B: \textit{drop blocks BELOW the block I just dropped}

B: \textit{But only 2}

A: \textit{how far down?}

B: \textit{directly below, and then just below that}

\strut[A drops block at (372,248)]
\strut[A drops block at (372,264)]

B: \textit{perfect}

\strut[B pushes ball upwards into teleporter at (380, 80)]

B: \textit{Tadaaa}

\hangindent=1.3em A: \textit{i see a ball.  on top of page all the way to the top but
  above right nostril}
}
\label{ex:instructions}
\end{example}
}

In Example \ref{ex:strategy}, player B is describing the plan he is
currently following to player A.


{\footnotesize
\begin{example}
\parbox[t]{0.9\columnwidth}{
B: \textit{So I need to get the ball into MY teleporter}

B: \textit{On the nose of the snake god.}

B: \textit{So you can see it, and then push it into the goal.}

B: \textit{Does that make sense?}

A: \textit{yes}
}
\label{ex:strategy}
\end{example}
}

Example \ref{ex:comparing} shows how the players notice a difference
in their maps. This is triggered by player A instructing player B to
drop a block in a location that is not accessible to player B.
At the end of the excerpt, the players exchange some chitchat followed
by messages that help to coordinate their communication.

{\footnotesize
\begin{example}
\parbox[t]{0.9\columnwidth}{
\hangindent=1.3em A: \textit{so, i'll place a block and you place a block to the right of it.}

[A drops a block at (104,20)]

\hangindent=1.3em B: \textit{Ok. *I* have a boundary of blocks extending down from the top to the bottom right corner of  the tomb of the island king}

B: \textit{and then across to the dead mans cove}

\hangindent=1.3em B: \textit{That whole area, containing the ruins, is inaccesible to me.}

B: \textit{Don;t YOU hate Kris now too?}

A: \textit{:-)}

[A drops a block at (104,20)]

A: \textit{slow down, i need to read what you wrote.}

B: \textit{I type smaller words.}
}
\label{ex:comparing}
\end{example}
}
