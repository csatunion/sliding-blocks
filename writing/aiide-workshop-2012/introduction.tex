
\section{Introduction}

The development of a natural language processing (NLP) applications
often requires human data. In particular, corpora of human language in
the target domain are an essential resource for designing, training,
and evaluating NLP systems, and task based evaluations where human
users interact with an NLP system give the most realistic assessment
of the system. Both corpus production and task based evaluations are
expensive and time-consuming, with one major factor being the
recruiting of participants.

Recently, a number of projects have used online games to recruit
people to help with the collection and annotation of corpora and the
task-based evaluation of NLP applications.  For example, in the
Restaurant Game\footnote{http://web.media.mit.edu/~jorkin/restaurant}
\cite{orkin-roy-2007} players play the roles of waitress and customer
in a restaurant. The resulting corpus of human language and actions in
this scenario was used to automate the construction of a
conversational character that could play one of these roles.

\cite{chamberlain-etal-2008} crowd-source the annotation of anaphoric
expressions in texts. Players/annotators in the game Phrase
Detective\footnote{http://anawiki.essex.ac.uk/phrasedetectives}
collect points for creating or validating annotations. They receive
titles when passing certain thresholds and can compare themselves to
other players on a leaderboard. 

The GIVE Challenge\footnote{Generating Instructions in Virtual
  Environments; http://give-challenge.org/research}
\cite{koller-etal-2010-give1-book,striegnitz-etal-2011-give25} invited
players to find a trophy in a virtual environment by following
automatically generated instructions. Players were paired with natural
language generation systems supplied by different research teams and
the collected data was used to evaluate and compare the systems.


\todo{What we want to do with this paper:
\begin{itemize}
\item this paper: continues GIVE line of research
\item situated language - action and language
\item more balanced
\item fun 
\item problems with GIVE: not fun, not a game
\end{itemize}
}

This paper describes a puzzle game in which two players collaborate to
push one or more balls into a goal position. The players are in two
different 2D-environments, but they can drop blocks into the other
environment, which are necessary to direct the ball. Furthermore,
there are portals which allow the balls to pass back and forth between
the environments. The players may or may not be able to see their
partner's environment, but they can communicate using a chat
interface.

The game was designed to ellicit a mixed-initiative, task-oriented
conversation in which both players alternate between the roles of
instruction giver and instruction follower. We are particularly
interested in situated language use and how actions in the game
environment contribute to the dialog.  By manipulating whether or not
the players can see each other's environment, we want to compare the
effects of a more directly shared workspace with a situation in which
the players receive only very indirect information about their
partner's environment (through the locations at which blocks are
dropped). Finally, we want the game to be fun. \todo{Why?}

\todo{another requirement: environment had to be simple enough that it
  can be modeled for an automated planner}
