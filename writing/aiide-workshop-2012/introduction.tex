
\section{Introduction}

The development of a natural language processing (NLP) applications
often requires human data. In particular, corpora of human language in
the target domain are an essential resource for designing, training,
and evaluating NLP systems, and task based evaluations where human
users interact with an NLP system give the most realistic assessment
of the system. Both corpus production and task based evaluations are
expensive and time-consuming, with one major factor being the
recruiting of participants.

Recently, a number of projects have used online games to recruit
people to help with the collection and annotation of natural language
corpora and the task-based evaluation of NLP applications.  For
example, in the Restaurant
Game\footnote{http://web.media.mit.edu/~jorkin/restaurant}
\cite{orkin-roy-2007} players assume the roles of a waitress and a
customer in a restaurant. The resulting corpus of human language and
actions in this scenario was used to automate the construction of a
conversational character that could play one of these roles.
\citeN{chamberlain-etal-2008} crowd-source the annotation of anaphoric
expressions in texts. Players/annotators in the game Phrase
Detective\footnote{http://anawiki.essex.ac.uk/phrasedetectives}
collect points for creating or validating annotations. They receive
titles when passing certain thresholds and can compare themselves to
other players on a leaderboard. 
The GIVE Challenge\footnote{Generating Instructions in Virtual
  Environments; http://give-challenge.org/research}
\cite{koller-etal-2010-give1-book,striegnitz-etal-2011-give25} invited
players to find a trophy in a virtual environment by following
automatically generated instructions. Players were paired with natural
language generation systems supplied by different research teams and
the collected data was used to evaluate and compare the systems.

This paper describes a puzzle game in which two players collaborate to
push one or more balls into a goal position. The players are in two
different 2D-environments, but they can drop blocks into the other
environment, which are necessary to direct the ball. Furthermore,
there are portals which allow the balls to pass back and forth between
the environments. The players may or may not be able to see their
partner's environment, but they can communicate using a chat
interface.

With this game we want to expand on the research started in the GIVE
project. As in GIVE, we are interested in the way humans interleave
language and actions when they are situated in an environment. This is
a topic that has recently started to receive increased attention in
the natural language generation and dialog systems communities, e.g.
\cite{stoia-etal-2006,garoufi-koller-2010,dethlefs-etal-2011}.  By
manipulating whether or not the players can see each other's
environment, we want to compare the effects of a shared environment
with situations in which the players receive only very indirect
information about their partner's environment (through the locations
at which blocks are dropped).

In GIVE there are two distinct roles -- the instruction giver, who can
send messages to the instruction follower but not act in the
environment, and the instruction follower, who can act in the
environment but not respond by sending messages. In the game described
in this paper, we want to create a more balanced scenario in which
both players can act as well as contribute to the chat
conversation. Furthermore, we want both players to sometime be in the
role of the instruction giver and sometimes in the role of the
instruction follower.


Finally, we want the game to be fun so that people play multiple
levels, return to play again, and tell their friends about it.  The
feedback we received from participants in the GIVE evaluations
indicates that while many players appreciated GIVE as a research
project, many others players were disappointed. They came to it
expecting a game but then discovered that all they had to do was
follow instructions. There were no puzzles to solve for the
instruction follower and no creative way to contribute to the solution
of the task. In fact, a significant number of players quit the game
before finishing, and most of the games that are canceled or lost end
quickly.

The game described in this paper currently exists as a prototype and
we are conducting playtests to refine the puzzles and the environment
in order to elicit the kinds of natural language interactions we are
interested in. The next section describes the game design. We then
sketch its implementation and give examples of the kind of data we are
collecting. We end by outlining our plans for the future.


